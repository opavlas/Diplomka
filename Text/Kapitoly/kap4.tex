\chapter{Realizace}
\label{realizace}
% promereni zapojeni jestli vse funguje 
% TODO + struktura firmware a provadeni digitalniho testu -> KAM TO ZARADIT? mozna do subsection realizace?
V této části bude popsána detailní realizace celého zařízení. V předchozích částech byly zmíněny základní požadavky \ref{Technicka specifikace} na návrh celého vyčítacího rozhraní a uveden základní koncept řešení \ref{Koncept reseni}. Při výběru individuálních částí rozhraní byly tyto části respektovány. Dále jednotlivé části byly vybírány s ohledem na miniaturizaci rozhraní a celkovou spotřebu. 

\par Pro další části textu bude označení desek plošných spojů navrženého rozhraní shodné, s označením schematického návrhu konceptu řešení z obrázku \ref{fig:navrh_reseni}. Cílem řešení bylo co nejvíce funkcionalit rozhraní implementovat na základní desce. Prvním důvodem bylo, že druhá deska plošných spojů - chipboard, obsahuje detektor Timepix 2, který je v celém návrhu nejdůležitější a nejsložitější částí. Pokud by bylo vše implementováno na jedné desce plošných spojů, při jakémkoliv problému musí být vyměněna celá deska i s detektorem Timepix 2. Přitom při rozložení rozhraní na dvě desky plošných spojů dojde při případném problému k výměně jen základní desky, popřípadě chipboardové desky. 
\par Druhým důvodem rozložení rozhraní na dvě desky plošných spojů je minimalizace rozměrů rozhraní. Za použití konektoru celé rozhraní zvýší své rozměry pouze na výšku o 3 mm přitom rozměry chipboardové desky mohou být stejné jako desky základní.

\par Realizace návrhu desek plošných spojů probíhala v programu Altium Designer. Mechanické integrace rozhraní, především validace případných kolizí plošných spojů s navrženou mechanikou, probíhala v programu Autodesk Inventor.

\section{Základní deska}	
	\label{zakladni deska}
	Základní deska je navržena na šesti vrstvém plošném spoji. Rozložení jednotlivých vrstev lze vidět na obrázku \ref{tab:pcb_vrstvy}. Celková tloušťka výsledného PCB je dle celkové skladby z obrázku \ref{fig:PCB_mother_stackup}, 1.6 mm. Vnější rozměry základní desky jsou 53 x 17 mm. Obecné části, které jsou na základní desce, jsou shodné s konceptem řešení \ref{fig:navrh_reseni}. Dále bude popsána detailní struktura jednotlivých částí základní desky.

\begin{figure}
	\begin{minipage}[b]{.45\linewidth}
		\centering
		\captionsetup{justification=centering}
		\includegraphics[scale=0.60]{PCB_mother_stackup.jpg}
		\caption{Rozložení vrstev PCB základní desky} 
		\label{fig:PCB_mother_stackup}
	\end{minipage}\hfill
	\begin{minipage}[b]{.45\linewidth}
		\centering
		\begin{tabular}{ |P{3cm}||P{3,5cm}|  }
			\hline
			\multicolumn{2}{|c|}{Rozložení vrstev PCB základní desky} \\
			\hline
			Vrstva  & Popis\\ \hline \hline 
			1 - TOP& Signálová vrstva\\ \hline		
			2 - GND1& Zemní vrstva \\ \hline 		 
			3 - SIG& Signálová vrsta \\ \hline
			4 - PWR& Napájecí vrstva\\ \hline
			5 - GND2& Zemní vrstva\\ \hline
			6 - BOT& Signálová vrstva\\ \hline
		\end{tabular}
		\caption{Popis vrstev PCB základní desky}
		\label{tab:pcb_vrstvy}
	\end{minipage}
\end{figure}

	\subsection{Napájení}
	Na základní desce je realizováno napájení, které je dále používáno pro celé výčítací rozhraní. Celkem jsou zde tři spínané synchronní step-down buck regulátory. Konkrétně se jedná o regulátor MP2333H \cite{MPH2333} od společnosti Monolithic Power Systems. Regulátor pracuje v rozsahu vstupních napětí on 4.2 - 18 V. Maximální výstupní proud jsou 3 A, spínací frekvence regulátoru je 1.2 MHz. Regulátor je možné pořídit v pouzdře SOT583, s rozměry 1.6x2 mm, které jsou pro úlohu miniaturizace rozhraní vyhovující. V tabulce \ref{tab:napajeni} můžete vidět seznam napájení generujících na základní desce.
	\begin{table}[h!]
		\centering
		\begin{tabular}{ |P{3cm}||P{10cm}|  }
			\hline
			\multicolumn{2}{|c|}{Napájení základní desky vyčítacího rozhraní} \\
			\hline
			Napájení  & Popis\\ \hline \hline 
			+5V & Externí napájení vyčítacího rozhraní přes USB typu C. \\ \hline		
			+3V3 & Napájení pro mikrokontrolér, CPLD a další 3.3V periférie \\ \hline 		 
			+2V5 & Napájení vstupní a výstupní brány detektoru Timepix 2 \\ \hline
			+1V2 & Napájení vstupní/výstupní brány CPLD.\\ \hline
		\end{tabular}
		\caption{Napájení základní desky vyčítacího rozhraní}
		\label{tab:napajeni}
	\end{table}
	Napájení uvedené v tabulce \ref{tab:napajeni} jsou až napájení 1.2 V, dostupné také na druhé desce plošných spojů - chipboardu. Více o propojení signálů základní desky a chipboardové desky lze dohledat v části textu \ref{konektor}. Ukázkové schematické zapojení jednoho ze tří spínaných regulátorů, můžete vidět na obrázku \ref{fig:mp2333h}. Konkrétně se jedná o zapojení, při kterém regulátor reguluje ze vstupních +5 V na napájení +1.2 V.
	\begin{figure}[h!]
		\centering
		\captionsetup{justification=centering}
		\includegraphics[scale=0.50]{mp2333h.pdf}
		\caption{Zapojení regulátoru MP2333H} 
		\label{fig:mp2333h}
	\end{figure}
	\subsubsection{Napájecí sekvence}
	Napájecí sekvenci je možné vidět na zjednodušeném diagramu na obrázku \ref{fig:napajeci_sekvecne}. Po připojení USB typu C do konektoru na základní dece je dostupné napájení +5 V. Těchto +5 V spíná první regulátor, který generuje na výstupu +3.3 V. Pokud je toto výstupní napájení +3.3 v pořádku, integrovaný obvod tuto informaci signalizuje pomocí pinu PG. Právě tento pin PG, signál PG\_3V3 může být je připojen na vstupní pin dalšího regulátoru a to na pin regulátoru generující výstupní napětí +2.5 V. V návrhu řešení je regulátor pro napájení +2.5 V, také možno řídit z mikrokontroléru. Poslední regulátor s výstupním napětím +1.2 V je řízený z mikrokontroléru, neboli jeho vstupní pin EN je propojen signálem EN\_1V2 s výstupní bránou mikrokontroléru.
	\begin{figure}[h!]
		\centering
		\captionsetup{justification=centering}
		\includegraphics[scale=0.80]{napajeci_sekvence.jpg}
		\caption{Napájecí sekvence základní desky} 
		\label{fig:napajeci_sekvecne}
	\end{figure}
	\par Výstupní napětí spínaného regulátoru je nastaveno pomocí napěťového děliče ve zpětné vazbě regulátoru a dáno vztahem dle \ref{eq:Vout}. Kde $V_{REF}$ = 805 mV.
	\begin{equation}
		V_{OUT} = \frac{R1 \cdot V_{REF}}{R2} + V_{REF}
		\label{eq:Vout}
	\end{equation}

	\par Z obrázku \ref{fig:napajeci_sekvecne} je vidět, že spínaný stabilizátor pro +3.3 V není programově ovladatelný z mikrokontroléru. Spínaný stabilizátor +3.3 V, z mikrokontroléru řídit nelze, protože právě těchto +3.3 V je napájením pro vybraný mikrokontrolér. Možností jakým mimo jiné zajistit dodržení vhodného časování napájecí sekvence základní desky, je propojení PG signálu stabilizátoru +3.3 V na signál EN stabilizátoru +2.5 V, nebo řízení stabilizátoru pro +2.5 V programově. Další možností rozfázování napájecí sekvence je volba vhodného kondenzátoru mezi pinem SS a zemním pinem ze zapojení \ref{fig:mp2333h}. Závislost velikosti výběru kondenzátoru na době rozběhu stabilizátoru je uvedena dle rovnice \ref{eq:Tss}.
	\begin{equation}
		T_{SS}\,[ms] = \frac{2V_{REF} \cdot C_{SS} \,[nF]}{I_{SS}}
		\label{eq:Tss}
	\end{equation}
	Kde $V_{REF}$ = 805 mV a $I_{SS}$ = 7.3 $\mu$A. Ze zapojení \ref{fig:mp2333h} a dosazení do vzorce \ref{eq:Tss} můžeme dopočítat, že rozběhový čas spínaného zdroje bude 1.4 ms. Ověření funkčnosti napájecí sekvence viz. \ref{fig:napajeci_sekvecne_real}. Kde CH1 je napájení +1.2 V, CH2 je +2.5 V a CH3 je vstupních +5V. Při tomto měření byl vstupní pin EN regulátoru pro +2.5 V propojen s výstupním pinem PG, regulátoru pro +3.3 V. Na nastavených kurzorech je také vidět, že teoreticky vypočítaný čas, doby rozběhu regulátoru z rovnice \ref{eq:Tss}, souhlasí s prakticky naměřenými hodnoty. 
	\begin{figure}[h!]
		\centering
		\captionsetup{justification=centering}
		\includegraphics[scale=0.50]{napajeci_sekvence_real.png}
		\caption{Měření napájecí sekvence základní desky} 
		\label{fig:napajeci_sekvecne_real}
	\end{figure}
	\par Zajištění ochrany, především před elektrostatickým výbojem vstupního napájení, bude popsána v části \ref{USB}. Pouze ve shrnutí, pokud dojde k jakýmkoliv podmínkám které by mohli elektricky ohrozit vyčítací rozhraní obvody z části \ref{USB} zajistí vypnutí externího napájení rozhraní pomocí externího tranzistoru. 

	\subsection{Mikrokontrolér}
	\label{mikrokontolér}
	Pro tuto práci byl vybrán mikrokontrolér od firmy STMicroelectronics, přesněji mikrokontrolér s označením STM32U5A9NJH6Q \cite{STM32U5A9}. Právě tento mikrokontrolér byl vybrán s ohledem na požadavky vyčítacího rozhraní, které byly uvedeny v předchozích částech textu. V následující části budou uvedeny nejdůležitější parametry vybraného mikrokontroléru:
	%\vspace{-5mm}
	\begin{itemize}
		\setlength\itemsep{0.005em}
		\item Jádro : Arm 32-bit Cortex-M33 s DSP a FPU. Frekvence 160 MHz
		\item Napájení 1.7 - 3.6 V
		\item Spotřeba 18.5 $\mu$A/MHz
		\item 4-Mbyte flash s kontrolou EEC
		\item 2514-Kbyte RAM, 66 Kbytes s EEC
		\item 25 Komunikačních periférií
		\item 156 konfigurovatelných vstupních/výstupních pinů
		\item 1 USB OTG high-speed s embedded PHY 
		\item Pouzdro : TFBGA216. 13 x 13 mm, 0.8 mm mezi pájecími plošky
	\end{itemize}
	\par Prvním požadavkem na mikrokontrolér bylo, aby bylo možné komunikovat přes sériovou datovou linku s detektorem Timepix 2. V již popsané části textu \ref{Komunikacni rozhrani}, bylo zmíněno, že Timepix 2, dokáže komunikovat po sériové lince s maximální frekvencí 100 Mhz. Výše uvedený vybraný mikrokontrolér umožňuje konfiguraci sériového komunikačního rozhraní, konkrétněji specfikace SPI až do frekvence 160 MHz. 
	\par Dalším důležitým parametrem při výběru mikrokontroléru byla velikost paměti. Pro vyčtení celé matice pixelů z detektoru Timepix 2, dle \ref{Digitálni cast} vyplývá, že je zapotřebí vyčíst 28 x 256 x 256 bitů dat, tedy 229.376 kB. Výše vybrané parametry pamětí mikrokontroléru jsou pro tento datový tok dostačující.
	\par Nejméně důležitým parametrem při výběru mikrokontroléru byl parametr integrovaného USB přímo uvnitř mikrokontroléru. Neboli není tedy zapotřebí při návrhu USB umísťovat další externí součástky pro implementaci USB komunikace. Tímto parametrem mikrokontroléru dokážeme výrazně snížit počet použitých součástek a tím i rozměry celého vyčítacího rozhraní.
	
	\subsubsection{Konfigurace mikrokontroléru}
	Pro práci s mikrokontrolérem jsem použil vývojové prostředí STM32CubeIDE dodávané od společnosti STMicroelectronics, která je výrobcem vybraného mikrokontroléru. Výhodou vývojového prostředí je přímočará grafická konfigurace celého mikrokontroléru s kombinací klasické programové konfigurace. Na obrázku \ref{fig:STM32CubeIde} můžete vidět příklad nakonfigurovaného mikrokotroléru STM32U5A9 v pouzdře TFBGA216. Tmavě zelené body mezi piny znamenají uživatelsky nastavené rozhraní daného pinů mikrokontroléru.
	\subsubsection{Napájení}
	Na obrázku \ref{fig:napajeni_stm32} můžete vidět potřebná napájení pro vybraný mikrokontrolér.  
	\begin{figure}[h!]
		\begin{subfigure}{0.5\textwidth}
			\centering
			\captionsetup{justification=centering}
			\includegraphics[scale=0.60]{STM32CubeIde.jpg}
			\caption{STM32CubeIDE konfigurace pinů mikrokontroléru} 
			\label{fig:STM32CubeIde}
		\end{subfigure}
		\begin{subfigure}{0.5\textwidth}
				\centering
			\captionsetup{justification=centering}
			\includegraphics[scale=0.60]{STM32_napajeni.jpg}
			\caption{STM32U5A9 napájení} 
			\label{fig:napajeni_stm32}
		\end{subfigure}
		\caption{Konfigurace a napájení STM32U5A9}
		\label{fig:konfig}
	\end{figure} 
	\begin{table}[h!]
		\centering
		\begin{tabular}{ |P{3cm}||P{10cm}|  }
			\hline
			\multicolumn{2}{|c|}{Napájení potřebné pro mikrokotrolér} \\
			\hline
			Napájení  & Popis\\ \hline \hline 
			VBAT & Napájení z externí baterie 1.65 - 3.6 V\\ \hline		
			VDDUSB & Napájení periférie USB\\ \hline 		 
			VDDSI & Napájení pro periféri DSI \\ \hline
			VDDSMPS & Napájení pro integrovaný spínaný stabilizátor\\ \hline
			VLXSMPS & Spínaný výstup integrovaného stabilizátoru \\ \hline
			VDD11 & Napájení digitální části mikrokotroléru \\ \hline 
			VDD11, VDD & Napájení digitální části mikrokotroléru ze spínaného stab.\\ \hline
			VDDIO2 & Napájení samostatné vstupní/výstupní brány \\ \hline
			VDDA & Napájení analogové části mikrokotroléru \\ \hline
		\end{tabular}
		\caption{Napájení mikrokontroléru STM32U5A9}
		\label{tab:napajeni_stm32}
	\end{table}
	Výhodou vybraného mikrokontroléru je, že pro napájení jádra a digitálních periferií využívá spínaného regulátoru integrovaného přímo na čipu, který má nižší spotřebu oproti lineárnímu regulátoru. Nevýhodou tohoto napájení je požadavek připojení externí cívky, která je nezbytná pro provoz interního spínaného stabilizátoru a tím tak větší požadavky na rozměry celého zapojení. Více podrobností ohledně napájení lze najít v referenčním manuálu viz. \cite{STM32U5A9_RM}. Celá realizace shematického zapojení napájení mikrokontroléru viz. přiložená příloha \ref{Priloha základní deska}.
	
	\subsubsection{Konfigurace hodinových signálů} %Pro USB HS plus konfigurace ostatnich periferii ?
	Hodinový signál označován jako SYSCLK, může být pro jádro mikrokontroléru jeden ze čtyř dostupných zdrojů hodinového signálu, kterými jsou:
	\begin{itemize}
		\setlength\itemsep{0.005em}
		\item HSE : Externí krystal s parametry frekvencí hodin od 4 MHz do 50 Mhz
		\item HSI : 16 MHz interní RC oscilátor 
		\item MSI : Interní RC oscilátor s programově nastavitelnou frekvencí od 100 kHz do 48 MHz
		\item PLL : Fázový závěs, který může mít vstup jeden ze 3 výše uvedených zdrojů hodinového signálu
	\end{itemize}
	Pro realizaci vyčítacího rohraní byl jako zdroj hodinového signálu vzbrán výstup z fázového závěsu, který ze vstupního hodinového signálu HSI generuje hodinový signál o frekvenci 160 MHz. Tento hodinový signál je použit jako zdroj hodinového signálu pro jádro mikrokontroléru a také pro komunikační rozhraní, až na komunikační rozhraní USB. Více o popisu konfigurace hodinového signálu pro USB periferii bude v této části \ref{USB}. Samotnou konfiguraci hodinového signálu pro jádro procesoru s využitím fázového závěsu, lze najít na obrázku \ref{fig:hodinovy_signal}
	\begin{figure}[h!]
		\centering
		\captionsetup{justification=centering}
		\includegraphics[scale=0.65]{hodinovy_signal.jpg}
		\caption{Konfigurace hodinového signálu pro jádro mikrokontroléru} 
		\label{fig:hodinovy_signal}
	\end{figure}

	\subsubsection{Konfigurace periférií SPI}
	Celkem pro implementaci vyčítacího rozhraní byly použity tři SPI periférie:
	 \begin{itemize}
	 	\setlength\itemsep{0.005em}
	 	\item SPI1 : Použito jako obecné SPI pro komunikaci se senzory, mimo Timepix 2
	 	\item SPI2 : Slouží ke komunikaci s Timepix 2. V konfigurace Master. 
	 	\item SPI3 : Slouží ke komunikaci s Timepix 2. V konfigurace Slave.
	 \end{itemize}
 	Sběrnice SPI1 je nakonfigurována v módu Full-Duplex Master, neboli po sběrnici je možné data z mikrokontroléru odesílat i přijímat. K určení jaké zařízení, připojené do sběrnice SPI1, budou komunikovat slouží signály chip select označován jako nCS (pozn.: prefix n, značí, že signál je aktivní v logické nule). 
 	\par Pro komunikaci s Timepix 2 pomocí sériového rozhraní, byly zvoleny dvě periférie SPI. Popis signálů pro sběrnice SPI2 a SPI3 můžete vidět na obrázku \ref{fig:SPI}. Jak lze ze zapojení vidět pro komunikace bylo využito celkem 6 signálů. Základní napěťové úrovně, které mají být na signálech, jsou určeny pull down, respektive pull up rezistory. % TODO dát to jinam ? Konkrétně ze zapojení \ref{fig:SPI} je vidět, že v návrhu byla použita odporová síť. Výhoda použití odporové sítě oproti diskrétních SMD rezistorů, je především v ušetření místa na desce plošných spojů. + uvest usetreni mista  
 	
 	%TODO Rozvinout ?? popsat casovani atd.. 
 	\par Sběrnice SPI2 je nakonfigurována jako Half-Duplex Master. Pomocí této sběrnice se odesílají data do detektoru Timepix 2. Dalšími důležitými parametry nastavení SPI sběrnice jsou její časování, které musí odpovídat technické specifikaci detektoru Timepix 2. S odkazem na tuto specifikaci \cite{Timepix2}, byla periferie SPI2 nastavena dle \cite{SPI} do módu 3. Neboli pro typ Half-Duplex Master to znamená, že data jsou vysílaná na sestupnou hranu hodinového signálu, přičemž klidový stav hodinového signálu je v logické jedničce.
 	
 	\par Sběrnice SPI3 je nakonfigurována jako typ Half-Duplex Slave. Pomocí této sběrnice se přijímají data z detektoru Timepix 2. Dle \cite{Timepix2} byla sběrnice nakonfigurována do módu 1 \cite{SPI}. příchozí data z detektoru Timepix 2 jsou vzorkována na sestupnou hranu hodinového signálu, který je společně s daty generován detektorem Timepix 2. Klidový stav hodinového signálu je poté v logické nule.
	\begin{figure}[h!]
 		\centering
 		\captionsetup{justification=centering}
 		\includegraphics[scale=0.40]{SPI.pdf}
 		\caption{Popis signálů periférií SPI, sloužící pro komunikaci s Timepix 2} 
 		\label{fig:SPI}
 	\end{figure}
 	
 	\subsubsection{Konfigurace periférií I2C}
 	Další sběrnicí, používanou vyčítacím rozhraním je sběrnice I2C. Pro celé rozhraní byla použita právě jedna I2C periférie. Pomocí této sběrnice mikrokotrolér monitoruje teplotu na druhé desce plošného spoje, chipboardu. Více o komunikaci s teplotním senzorem viz. \ref{Mereni teploty}. Pokud by I2C sběrnici bylo připojeno více senzorů, určení toho, kdo po sběrnici bude přijímat data závisí na fyzické adrese připojeného zařízení. V tomto případě ale musí být zajištěno, že na sběrnici neexistují dvě zařízení se stejnou fyzickou adresou. 
 	
 	\subsubsection{Konfigurace periférií UART}
 	\par Nakonfigurovaná periférie UART umožňuje monitorovat průběh programu, který je nahrán do mikrokontroléru, za použití výpisu přes sériovou linku, do příkazové řádky. Signály UART sběrnice jsou vyvedeny na programovací konektor, více viz následující část.

%TODO zminit literaturu ZAHLAVA A MCU
	\subsubsection{Programování}
	Mikrokontrolér je programován pomocí JTAG rozhraní za využití programovacího konektoru, který je umístěn na spodní straně základní desky \ref{fig:programator}. Jedná se o 12 pinový konektor pro plochý kabel. Na programování mikrokontroléru byly z mikrokontroléru vyvedeny signály: JTMS/SWDIO a JTCK/SWCLK. Zbylé piny konektoru byly využity pro programování MachXO2 \cite{CPLD}. Tedy přes plochý konektor z obrázku \ref{fig:programator} lze progrmovat jak mikrokontolér, tak na základní dece použité CPLD: MachXO2 \ref{CPLD}. Posledními signály vyvedenými na programovací konektor jsou signály pro monitorování běhu programu, jedná se o signály komunikačního rozhraní sběrnice UART. Signály jsou za pomocího plochého kabelu vyvedeny z plochého konektrou z obrázku \ref{fig:programator} na redukční desku, na které je možné pomocí propojovacích kabelů připojit použitý programátor ST-Link, právě s navrženou konverzní deskou. Toto jednoduché PCB můžete vidět na obrázku \ref{fig:konverzni_deska}. 
\begin{figure}[h!]
	\begin{subfigure}{0.5\textwidth}
		\centering
		\captionsetup{justification=centering}
		\includegraphics[scale=0.48]{programator.jpg}
		\caption{Programovací konektor} 
		\label{fig:programator}
	\end{subfigure}
	\begin{subfigure}{0.5\textwidth}
		\centering
		\captionsetup{justification=centering}
		\includegraphics[scale=0.22]{konverzni_deska.jpg}
		\caption{Konverzní PCB pro připojení ST-Link programátoru} 
		\label{fig:konverzni_deska}
	\end{subfigure}
	\caption{Programování mikrokontroléru}
	\label{fig:programovani}
\end{figure} 

	\subsubsection{PCB}
	Realizované PCB pro část mikrokontroléru můžete vidět na obrázku \ref{fig:ST_layout}. Na obrázku nejsou uvedeny 3 zbylé vrstvy PCB, kterými jsou 2., 3. a 5. vrstva. Vrstva 2 a 5 tvoří pod částí mikrokontroléru souvislou zemní plochu a vrstva 3 slouží jako vrstva signálová. Blokovací kondenzátory napájení dle specifikací \cite{STM32U5A9_RM} jsou umístěny co nejblíže příslušným pinům, z druhé strany, než je samotný mikrokontrolér, viz. \ref{fig:ST_BOT}. 
	\begin{figure}[h!]
		\centering
		\captionsetup{justification=centering}
		\begin{subfigure}[b]{0.3\textwidth}
			\centering
			\includegraphics[scale=0.30]{ST_TOP.jpg}
			\caption{Vrchní vrstva PCB STM32U5A9}
			\label{fig:ST_TOP}
		\end{subfigure}
		\hfill
		\begin{subfigure}[b]{0.3\textwidth}
			\centering
				\includegraphics[scale=0.30]{ST_PWR.jpg}
			\caption{Prostřední vrstva PCB STM32U5A9}
			\label{fig:ST_PWR}
		\end{subfigure}
		\hfill
		\begin{subfigure}[b]{0.3\textwidth}
			\centering
				\includegraphics[scale=0.30]{ST_BOT.jpg}
			\caption{Spodní vrstva pcb STM32U5A9}
			\label{fig:ST_BOT}
		\end{subfigure}
		\caption{STM32U5A9 PCB realizace}
		\label{fig:ST_layout}
	\end{figure}
	Rozteč mezi jednotlivými piny mikrokontroléru je 0.8 mm. Jak lze vidět na obrázku \ref{fig:ST_TOP}, mezi pájecími plošky jsou umístěny prokovy. Vnitřní průměr prokovů je 0.3 mm a vnější rozměr 0.4 mm. Prokovy jsou kvůli zvolenému pouzdru BGA mikrokontroléru zamaskované a vyplněné epoxidem, aby při procesu pájení nedošlo ke zkratování prokovů s piny mikrokontroléru.
	\par Osazení mikrokontroléru proběhlo na půdě ČVUT v laboratoři LVR (Laboratoř pro Vývoj a Realizaci). Pro osazení byl použita rework stanice: ERSA - IRPL650A.
	
	\subsection{CPLD}	% CPLDcko - delice, urovne. Odporove site
	\label{CPLD}
	Dlaší hlavní součástí základní desky, je CPLD. Pro tuto práci bylo konkrétně vybráno CPLD od společnosti Lattice Semiconductor s označením MachXO2 \cite{CPLD}. Jak bude popsáno v následné části, CPLD je použito pro konverzi logických úrovní a generování diferenciálních signálů které jsou potřebné pro komunikaci s detektorem Timepix 2. S ohledem na tyto požadavky bylo vybráno právě CPLD MachXO2 v pouzdře QFN-48. Programování CPLD probíhalo v prostředí Lattice Diamond.
	
	\subsubsection{Napájení}
	V tabulce \ref{tab:napajni_cpld} můžete vidět napájecí úrovně pro MachXO2. Pro napájení jádra CPLD a vstupních/výstupních bran 0, 1 a 2 je použito napájení +3.3V. Pro vstupní/výstupní bránu 3, je napájecí napětí +1.2 V. Toto napájení bylo zvoleno s ohledem na použití této brány jako typ LVCMOS12D, pro generování diferenciálních páru odpovídající specifikaci SLVS \ref{fig:SLVS_LVDS}.
	\begin{table}[h!]
		\centering
		\begin{tabular}{ |P{3cm}||P{5cm}||P{3cm}|  }
				\hline
			\multicolumn{3}{|c|}{Napájení CPLD MachXO2} \\
			\hline
			Napájení  & Popis & Hodnota [V]\\ \hline \hline 
			VCC & Hlavní napájení jádra CPLD & +3.3 V\\ \hline		
			VCCIO0, 1, 2 & Napájení brány 0, 1, 2 & +3.3V\\ \hline 		 
			VCCIO3 & Napájení brány 3 & +1.2 V \\ \hline
		\end{tabular}
		\caption{Napájení CPLD MachXO2}
		\label{tab:napajni_cpld}
	\end{table}

	\subsubsection{Konverze logických úrovní}
	Dle \ref{Komunikacni rozhrani}, je zapotřebí pro komunikaci s Timepix 2 při využití sériového rozhraní použít komunikační specifikaci typu SLVS \ref{fig:SLVS_LVDS}, využívající diferenciální páry. Použití mikrokontrolér \ref{mikrokontolér}, je kompatibilní pouze s CMOS 3.3 V logikou. Použité CPLD umožňuje převod těchto úrovní. Respektive převod CMOS 3.3 V logiky na SLVS specifikaci a opačně.
	\par Použité CPLD umožňuje generovat pouze signály ze specifikace Sub-LVDS. Aby bylo dosaženo parametrů potřebných pro SLVS specifikaci, bylo pro každý generovaný výstupní diferenciální pár použito zapojení s odpory z pravé části obrázku \ref{fig:sub_lvds_slvs}. Pro vstupní diferenciální páry, byla pouze co nejblíže k CPLD umístěna terminace, viz. levá část obrázku \ref{fig:sub_lvds_slvs}.
	\begin{figure}[h!]
		\centering
		\captionsetup{justification=centering}
		\includegraphics[scale=0.50]{sub_lvds_slvs.pdf}
		\caption{Konverze Sub-LVDS na SLVS} 
		\label{fig:sub_lvds_slvs}
	\end{figure}
	Programová implementace firmware pro MachXO2 obsahující pouze část architektury pro konverzi logických úrovní je zobrazena v ukázkovém kódu \ref{kod_vhdl}.
\begin{lstlisting}[frame=single, language=VHDL, caption={Ukázkový kód ve VHDL pro CPLD}, label=kod_vhdl]
architecture behavioral of t2m is
begin
-- CMOS -> SLVS
DATA_IN_SLVS	<= DATA_IN;
nCS_IN_SLVS	<= nCS_IN;
DCLOCK_IN_SLVS	<= DCLOCK_IN;
MCLOCK_SLVS	<= MCLOCK;
-- SLVS -> CMOS
DATA_OUT	<= DATA_OUT_SLVS;
DCLOCK_OUT	<= DCLOCK_OUT_SLVS;
nCS_OUT		<= nCS_OUT_SLVS;
end architecture behavioral ;
\end{lstlisting}
	\subsubsection{Programování}
	Pro programování byla zvolena dvojí možnost. Primární možností je programování CPLD, přes externí programovací konektor, který se také používá pro programování mikrokontroléru \ref{fig:programator}. Na tento konektor jsou vyvedeny signály: TDO, TDI, TMS a TCK. Druhou možností je programování CPLD pomocí mikrokontroléru za využití SPI sběrnice. Výhodou možnosti programování CPLD z mikrokontroléru je možnost měnit konfiguraci CPLD za běhu programu. V této realizaci je tato možnost pouze připravena pro  případné budoucí využití. S ohledem na využití CPLD v této práci jako převaděč úrovní, není zapotřebí jakkoliv překonfigurovávat CPLD po nahrání programu přes JTAG rozhraní.
	\subsubsection{Využití odporových sítí} % TODO davat to sem, nebo nekam? Je to ohledne miniaturizace mozna dobre zminit, ze se slo do detailu aby se usetrilo misto
	Jak bylo možné vidět na obrázcích \ref{fig:sub_lvds_slvs}, nebo \ref{fig:SPI}, pro zapojení s odpory byly použity odporové sítě. Výhodou použití odporových sítí je především možnost ušetření místa na desce plošných spojů, oproti použití diskrétních odporů. Ušetření místa za použití odporových sítí můžete vidět na obrázku \ref{fig:odpor_sit}, kde je porovnána odporová síť se čtyřmi odpory vůči použití čtyř diskrétních odporů v pouzdře 0402.
	\begin{figure}[h!]
		\centering
		\captionsetup{justification=centering}
		\includegraphics[scale=0.60]{odpor_sit.jpg}
		\caption{Odporová síť v porovnání s diskrétními odpory} 
		\label{fig:odpor_sit}
	\end{figure}
	
	\subsection{USB} % Navrh plosnaku, USB C ochrana, popis USB C konektoru
	\label{USB}
	Jako uživatelské rozhraní bylo zvoleno rozhraní USB s konektorem typu USB C. Konkrétně byla použita specifikace USB 2.0 High Speed. Tato specifikace umožňuje komunikovat s maximální přenosovou rychlostí 480 Mbit/s. Vzhledem k maximální vyčítací rychlosti z detektoru Timepix 2, která je dle \ref{Technicka specifikace} 100 Mbit/s, je použití této konkrétní specifikace dostačující. 
	\subsubsection{Konektror}
	Pro tuto práci byl použit konektor typu USB C. Jednou z výhod konektoru USB C je symetrické zapojení pinů na konektoru. Díky této vlastnosti může uživatel konektor zapojit do vyčítacího rozhraní s libovolnou orientací. Symetrické zapojení signálu na USB C konektoru je možné vidět na obrázku \ref{fig:USBC_full}. K detekci orientace zapojení konektoru, slouží signály označené jako CC1 a CC2. Další funkce těchto signálů je, že pomocí nich je možné nastavit konkrétní velikost proudu dodávaného do zařízení. 
	\begin{figure}[h!]
		\centering
		\captionsetup{justification=centering}
		\includegraphics[scale=0.50]{USBC_full.jpg}
		\caption{Konektor USB C} 
		\label{fig:USBC_full}
	\end{figure} 

	\subsubsection{Zapojení}
	Vzhledem k použité specifikaci USB 2.0, je možné některé piny ze zapojení konektoru \ref{fig:USBC_full} nezapojit. Konkrétně to jsou piny s označením TXxx a RXxx, tyto piny slouží pro přenos vysokorychlostní komunikace z USB standartu 3.0 a vyšší. Celkové zapojení použitého konektoru a potřebných elektro statických ochran je možné vidět na obrázku \ref{fig:USBC_zapojeni}. Pro elektorostatickou ochranu diferenciálních datových párů D+ a D- byla použita součástka ECMF02-2AMX6. Dalším prvkem pro ochranu USB C signálů byla použita součástka TCPP01-M12. Tato součástka zajišťuje ochranu před přepětím na napájecím napětí VBUS. Dále zajišťuje ochranu pinů CC před případných zkratem s napájecím napětí. Odporovým děličem ze zapojení \ref{fig:USBC_zapojeni}, který je tvořený odpory R34 a R36, je možné nastavit maximální hodnotu napájecího napětí, které bude dodáváno na výstupu pinu VBUS. V této práci je nastaveno maximální napájecí napětí na +6 V. V případě překročení této nastavené úrovně dojde pomocí externího tranzistoru Q1, k odpojení napájecího napětí zařízení. Jedinou částí vyčítacího rozhraní, ke které bude mít uživatel fyzický přístup je právě USB C konektor, proto bylo zvoleno zapojení ochrany, v podobě paraelního zapojení odporu R37 a kondenzátoru 56, na stínění konektoru USB C. 
	\begin{figure}[h!]
		\centering
		\captionsetup{justification=centering}
		%\includepdf[scale=0.7]{usbc.pdf}
		\includegraphics[scale=0.60]{usbc.pdf}
		\caption{Zapojení konektoru USB C} 
		\label{fig:USBC_zapojeni}
	\end{figure} 
	
	\subsubsection{PCB}
	Jak bylo zmíněno v předchozí části, maximální komunikační rychlost specifikace USB 2.0 High Speed může bít 480 Mbit/s. Proto návrh PCB musel odpovídat požadavkům pro návrh PCB pro vysokorychlostní komunikaci. Délka diferenciálních datových párů D+ a D- byla vůči sobě vykompenzována na vzdálenost 1 mm. Dle technické specifikace USB 2.0 High Speed je doba náběžné ($t_r$), respektive sestupné ($t_f$) hrany 500 ps. Časové zpoždění ($\Delta t$) mezi datovými páry D+ a D- by mělo být výrazně menší, než doba ($t_r$), respektive ($t_f$). Pokud zvolíme maximální časové zpoždění mezi páry D+ a D- na hodnotu 25 ps, při hodnotě zpoždění 0.1 ns/cm na DPS, je nutné diferenciální signály vykompenzovat, tak aby maximální rozdíl délek vodičů diferenciálních párů, nebyl větší než 2.5 mm. Jak bylo již zmíněno, v uvedené práci došlo ke délky diferenciálních párů na 1 mm, tedy šasové zpoždění mezi páry, nebude větší než 10 ps.
	\par Diferenciální páry D+ a D- byly po PCB vedeny ve vrchní a spodní vrstvě. Dle celkové skladby PCB základní desky z tabulky \ref{tab:pcb_vrstvy} odpovídá, že sousedící vrstva, respektive vrstvy pod vrchní a spodní vrstvou je vždy souvislá zemní plocha. Z celkové skladby PCB z obrázku \ref{fig:PCB_mother_stackup}, byly výrobcem PCB dodané parametry pro fyzické rozměry diferenciálních vodičů aby došlo k dodržení impedančního přizpůsobení $100 \Omega$. Realizaci vedení signálů D+ a D- na desce plošných spojů je možné vidět na obrázku \ref{fig:usb_layout}
	
	\begin{figure}[h!]
		\begin{subfigure}{0.5\textwidth}
			\centering
			\captionsetup{justification=centering}
			\includegraphics[scale=0.33]{usb_hs_top.jpg}
			\caption{Signály D+ a D-, vrchní vrstva PCB} 
			\label{fig:usb_top}
		\end{subfigure} \hfill
		\begin{subfigure}{0.5\textwidth}
			\centering
			\captionsetup{justification=centering}
			\includegraphics[scale=0.33]{usb_hs_bot.jpg}
			\caption{Signály D+ a D-, spodní vrstva PCB} 
			\label{fig:usb_bot}
		\end{subfigure}
		\caption{Vedení diferenciálních datových páru USB}
		\label{fig:usb_layout}
	\end{figure} 
%TODO kde popisu cely cyklus vycitani? v Testovani?
% TODO cite zahlava
\section{Deska s Timepix 2}
	Druhou deskou plošných spojů v této práci je deska označovaná jako chipboard ze schématického obrázku konceptu řešení \ref{fig:navrh_reseni}. Tato deska je propojena se základní deskou za pomocí 60 pinového konektoru ze série DF40 od společnosti Hirose Electric Group. Rozložení signálů na konektoru lze najít v příloze zapojení celé práce \ref{Priloha základní deska}.
	
	\par Deska s Timepix 2 je navržena na šestivrstvém plošném spoji. Skladba plošného spoje je analogická, jako skladba PCB pro základní desku viz obrázek \ref{fig:PCB_mother_stackup} a tabulka \ref{tab:pcb_vrstvy}. Rozměry PCB jsou poté 53 x 17 mm. 
\subsection{Timepix 2}	% 
	Nejduležitější částí celé práce je samotný pixelový detektor radiace Timepix 2. Specifikace pixelového detektoru Timepix 2 byly popsány v části \ref{Technicka specifikace}. Nadále bude popsána realizace pro splnění všech specifikacích nutných pro provoz pixelového detektoru Timepix 2.
	\subsubsection{Napájení}	% popsat jak se bere z konektoru plus prepinani pro chipID
	Dle \ref{tab:tpx2_napajeni} je pro korektní provoz Timepix 2 zapotřebí celkem třech napájecí napětí. Napájecí napětí pro vstupní/výstupní bránu detektoru Timepix 2 označovanou jako VDDIO s velikostí napájecího napětí +2.5 V se generuje na základní desce. Tedy toto napájení je vedeno přes konektor ze základní desky a je poté na desce chipboard přivedeno k příslušným pinům Timepix 2.
	\par Napájení pro digitální a analogovou část Timepix 2 se generuje přímo na chipboardové desce. Z označení z tabulky \ref{tab:tpx2_napajeni} se jedná o napájení VDD a VDDA. Zapojení regulátorů pro tyto napájecí napětí můžete vidět na obrázku \ref{fig:tpx_pwr}. Pro napájení digitální části Timepix 2 byl použit stejný spínaný regulátor, jako na základní desce \ref{fig:mp2333h}. Pro napájení analogové části byl použit lineární stabilizátor od společnosti Texas Instruments TPS7A94, který ze vstupních +2.5 V reguluje napětí na výstupních +1.2 V. Oba uvedené regulátory jsou řízené z mikrokontroléru, který je umístěn na základní desce \ref{zakladni deska}.
	
	\begin{figure}[h!]
		\centering
		\captionsetup{justification=centering}
		\includegraphics[scale=0.60]{tpx_pwr.pdf}
		\caption{Analogové a Digitální napájení Timepix 2} 
		\label{fig:tpx_pwr}
	\end{figure} 
	
	\subsubsection{Rozhraní pro připojení Timepix 2}	% wirebondovaci plosky, HV zdroj
	Pro připojení Timepix 2 k desce plošných spojů byly použity wirebondovcí plošky na vrchní straně Timepix 2, které je možné vidět na spodní straně obrázku \ref{fig:tpx2_floorplan}. Celekm bylo u této práce realizovat propojení 128 wirebondovacích plošek z Timepix 2 na chipboardovou desku. Plošky na straně PCB jsou umístěny ve dvou řadách. Rozteč mezi jednotlivými plošky je 0.2mm, šířka plošek je poté 0.1 mm. V pravé části obrázku \ref{fig:tpx_wire} můžete vidět plošku pro připojení vysokého napětí na senzorovou vrstvu Timepix 2. Toto vysoké napětí slouží k vytvoření vyprázdněné oblasti v senzorové vrstvě, více viz. \ref{kap:2.1}. 
	\begin{figure}[h!]
		\centering
		\captionsetup{justification=centering}
		\includegraphics[scale=0.60]{tpx_wire.png}
		\caption{Rozložení wirebondovacích plošek k připojení Timepix 2} 
		\label{fig:tpx_wire}
	\end{figure} 
	
		
	\subsection{Vysokonapěťový zdroj}	% MAX1932 zapojeni vysvetleni
	Vysokonapěťový zdroj pro tuto práci byl realizován za použití součástky od společnosti Analog Devices, konkrétně byla použita součástka MAX1932. Jedná se o vysokonapěťový zdroj s nastavitelnou úrovní výstupního napětí 30 - 148 V. Úroveň výstupního napětí se nastavuje pomocí SPI komunikace v uvedeném rozsahu s rozlišením 256 hodnot. Schematické zapojení vysokonapěťového zdroje je možné vidět na obrázku \ref{fig:hv}. Hodnota velikosti výstupního napětí lze nastavit pomocí odporů ve zpětné vazbě regulátoru ze zapojení \ref{fig:hv}. 
	
	\begin{equation}
		R5 = (V_{OUT01} - V_{OUTFF})\cdot (R6/1.25V)
		\label{eq:R5}
	\end{equation}

	\begin{equation}
		R6 = \frac{1.25V \cdot R5}{V_{OUTFF}}
		\label{eq:R6}
	\end{equation}
	
	\begin{figure}[h!]
		\centering
		\captionsetup{justification=centering}
		\includegraphics[scale=0.60]{hv.pdf}
		\caption{Zapojení vysokonapěťového zdroje MAX1932} 
		\label{fig:hv}
	\end{figure} 
		\subsubsection{Měření vysokého napětí} % napetovy sledovac plus konfigurace AD v STM32
	\subsection{Měření teploty}	% proces nastaveni senzoru atd
	\label{Mereni teploty}
	\subsection{Konektor}	% popis konektoru, zarazeni zmneni mezi piny
	\label{konektor}

