\chapter{Úvod}

% + žvásty o používání pixelových detekrorů -> nahnat reference

%TODO zminit UTEF, tady nebo v abstraktu?

Cílem této diplomové práce byla realizace miniaturizovaného vyčítacího rozhraní pro pixelový detektor radiace Timepix 2. Realizace diplomové práce probíhala na půdě Ústavu Technické a Experimentální Fyziky ČVUT v Praze. Text diplomové práce je rozdělen do následujících čtyř hlavních bloků.

\par \textbf{První část} práce se týká pixelových detektorů radiace a vyčítacích zařízení \ref{kap:2}. V této kapitole bude popsán obecný princip činnosti pixelových detektorů, především generace detektorů z rodiny Timepix, jež je vyvíjena pod záštitou CERN Medipix Collaboration \cite{Medpix}. Dále zde bude uveden stručný seznam vyčítacích rozhraní, která jsou nezbytnou součástí pro vyčítaní dat z detektorů Timepix.

\par \textbf{Druhá část} práce obsahuje koncept návrhu řešení \ref{Koncept reseni} této diplomové práce. V této části je diskutováno, jaké parametry vyčítacího rozhraní musí být splněny, aby byla zajištěna základní funkcionalita navrženého rozhraní.

\par \textbf{Třetí část} textu obsahuje podrobný popis realizace miniaturizovaného vyčítacího rozhraní pro pixelový detektor Timepix 2 \ref{realizace}. Podkapitoly této části jsou rozděleny do logických bloků, jež odpovídají samotným blokům skutečné realizace. Vedle schematických zapojení jednotlivých částí realizace, lze zde dohledat parametry jednotlivých částí realizovaného rozhraní.

\par \textbf{Čtvrtá část} je tvořena testováním navrženého rozhraní a ověřením funkcionality rozhraní jakou celku. Je zde popsáno testování jednotlivých funkčních celků, tak celého rozhraní. Komplexním testem pro navržené vyčítací rozhraní je poté připojení rozhraní do programu TrackLab \cite{Manek_2024}, které zde bude také popsáno.