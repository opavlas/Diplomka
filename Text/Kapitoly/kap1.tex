\chapter{Úvod}

% + žvásty o používání pixelových detekrorů -> nahnat reference

%TODO zminit UTEF, tady nebo v abstraktu?

Cílem této diplomové práce byla realizace miniaturizovaného vyčítacího rozhraní pro pixelový detektor radiace Timepix 2. Realizace diplomové práce probíhala na půdě Ústavu Technické a Experimentální Fyziky ČVUT v Praze. Text diplomové práce je rozdělen do následujících čtyř hlavních částí.

\par \hyperref[kap:2]{První část} práce se týká pixelových detektorů radiace a vyčítacích zařízení. V této kapitole bude popsán obecný princip činnosti pixelových detektorů, především generace detektorů z rodiny Timepix, jež je vyvíjena pod záštitou CERN Medipix Collaboration \cite{Medpix}. Dále zde bude uveden stručný seznam vyčítacích rozhraní, která jsou nezbytnou součástí pro vyčítaní dat z detektorů Timepix.

\par \hyperref[Koncept reseni]{Druhá část} práce obsahuje koncept návrhu řešení této diplomové práce. V této části bude diskutováno, jaké parametry vyčítací rozhraní musí splňovat, aby byla zajištěna základní funkcionalita navrženého rozhraní.

\par \hyperref[realizace]{Třetí část} textu obsahuje podrobný popis realizace miniaturizovaného vyčítacího rozhraní pro pixelový detektor Timepix 2. Podkapitoly této části jsou rozděleny do logických bloků, jež odpovídají samotným blokům skutečné realizace. Vedle schematických zapojení jednotlivých částí realizace, lze zde dohledat parametry jednotlivých částí rozhraní.

\par \hyperref[testovani]{Čtvrtá část} je tvořena testováním navrženého rozhraní a ověřením funkcionality rozhraní jako celku. Je zde popsáno testování jednotlivých funkčních celků, tak celého rozhraní. Komplexním testem pro navržené vyčítací rozhraní je poté připojení rozhraní do programu TrackLab \cite{Manek_2024} a obsluha rozhraní za využití právě tohoto programu. Finálním funkčním testem popsaným v této části je samotné měření s navrženým miniaturizovaným vyčítacím rozhraní, které bylo realizováno v laboratoři CERN.