\thispagestyle{plain}

\begin{center}
	\Large
	\textbf{Abstract}
\end{center}
This thesis focuses on the design of a miniaturized readout interface for the Timepix 2 pixel radiation detector. In the theoretical part of the thesis, the Timepix 2 pixel radiation detector is described together with the readout interfaces, which are an necessary part for the operation of the detectors. In the practical part the implementation of the designed interface is described in detail. Finally, the functionality of the designed interface is then demonstrated, with the final test being a functional measurement at the particle accelerator at CERN.

%\textbf{Keywords:} keyword.

\hfill

\begin{center}
	\Large
	\textbf{Abstrakt}
\end{center}

Tato diplomová práce se zabývá návrhem miniaturizovaného vyčítacího rozhraní pro pixelový detektor radiace Timepix 2. V teoretické části práce je uveden popis pixelových detektorů radiace Timepix 2 spolu s vyčítacími rozhraními, které jsou nezbytnou součástí pro provoz detektorů. V praktické části je popsána detailně realizace navrženého rozhraní. Na závěr je poté demonstrována funkčnost navrženého rozhraní, kdy finálním testem bylo funkční měření na urychlovači částic v laboratoři CERN.

